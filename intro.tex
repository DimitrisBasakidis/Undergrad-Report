\section{Introduction}
Modern high-performance full-text search engines like Apache Lucene \cite{lucene} are increasingly used in large-scale data processing and indexing. Lucene, which runs on the JVM, relies on DRAM to store indexing structures, field and query caches, query evaluation data to deliver low-latency performance. However, data volumes continues to expand at a rapid rate, also growing memory requirements accordingly. This presents a challenge, as improvements in DRAM capacity have not kept up with the rate of data growth in recent years \cite{DAOS, Borg}. As a result, the memory footprint of Lucene applications often becomes a bottleneck, particularly working with large indexes or real-time search workloads. The JVM heap and related managed memory components grow more slowly than the datasets Lucene is expected to handle, leading to pressure on memory management and performance tuning. In our experiments, we use TeraHeap as the underlying managed runtime system. TeraHeap introduces a dual heap design that places the primary heap (H1) in DRAM and a secondary heap (H2) on slower, larger-capacity memory such as SSD-backed storage. This separation enables the system to confine garbage collection (GC) activity to H1, avoiding the costly scanning and compaction of objects stored in the slow tier (H2). TeraHeap also uses a portion of DRAM as an I/O cache to speed up accesses to H2, allowing frequently accessed objects to benefit from DRAM latency.

In our setup, we use TeraHeap \cite{TeraHeap} as the runtime system and focus exclusively on varying the size of H1, the primary heap located in DRAM and managed by the G1 Garbage Collector. By increasing or decreasing the size of H1 at runtime, we evaluate how memory pressure affects application performance and garbage collection behavior. Since DRAM is shared between H1 and the page cache used to accelerate H2 accesses, changing the size of H1 effectively changes the partitioning of DRAM between managed memory and the cache for the slow-tier heap (H2). This setup allows us to study the trade-off between GC-managed heap size and available cache capacity for H2, and its impact on overall system performance.

In this work, we introduce a Dynamic Heap Resizer, which is a system that automatically decides at runtime whether to adjust the size of H1 or the H2 page cache. A Dynamic Heap Resizer is the first approach that enables practical and efficient deployment of hybrid heaps. Its design, which continuously rebalances DRAM allocation between H1 and the H2 cache, addresses some challenges in hybrid memory management.

Hybrid heap systems face significant challenges in effectively dividing a fixed DRAM budget between H1, the GC-managed heap, and the H2 page cache, which accelerates access to slow-tier memory. First, static DRAM partitioning leads to imbalanced performance trade-offs: (1) allocating too little memory to H1 results in high garbage collection overhead, while a small H2 cache increases I/O latency for accessing off-heap objects. (2) Existing approaches lack adaptability, as they require prior knowledge of application behavior or manual tuning to set optimal heap sizes. This is impractical for real-world applications with dynamic memory demands that change over time. (3) Even when DRAM allocations are adjusted, the impact of these changes is delayed due to OS memory reclamation and page cache resizing latencies, making current systems slow to react to shifting workload phases and limiting their responsiveness.

With the use of a Dynamic Heap Resizer, by continuously monitoring both garbage collection and I/O costs, decisions are made to resize H1 or adjust the effective cache space for H2, optimizing performance across diverse workload conditions. It employs an advanced adaptation mechanism based on a finite state machine (FSM) with wait states and direct transitions, enabling fast and stable adjustments as application memory requirements evolve. This approach ensures that DRAM is allocated efficiently between managed heap and page cache without requiring application-specific knowledge or manual intervention, improving overall system performance and responsiveness.


