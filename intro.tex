\section{Introduction}

% \note{jk: the first two paragraphs should describe the problem. 1. Applications
% need large heaps but DRAM does not scale. The second paragraph should describe
% that existing solutions like TeraHeap extend the heap over storage device but
% they have the DRAM division problem.}
%
Modern high-performance applications, such as full-text search engines like Apache Lucene 
\cite{lucene_dimitris}, increasingly operate on large-scale datasets to deliver low-latency 
search and analytics. Lucene relies heavily on DRAM to store its indexing structures, field 
and query caches, and evaluation data to achieve high throughput and fast response times. 
However, while data volumes continue to expand rapidly, DRAM capacity improvements have stagnated 
in recent years \cite{DAOS, Borg}. This growing imbalance creates a bottleneck, as Lucene’s memory 
requirements outpace available DRAM resources, leading to degraded performance, increased GC overheads, 
and complex memory management challenges for developers.

To address these limitations, TeraHeap \cite{TeraHeap} introduces a dual heap design,
allocating a primary heap (H1) for GC-managed short-lived objects and a secondary heap (H2) on 
SSD-backed storage where longer-lived data structures such as the Lucene query cache are placed to reduce GC overhead. 
Additionally, it uses a portion of DRAM as an I/O cache to accelerate accesses to H2 and hide storage latency. However, TeraHeap statically partitions 
DRAM between H1 and the page cache for H2, lacking dynamic adaptability to changing application memory pressure 
and I/O demands. This static division can lead to suboptimal performance, as the ideal DRAM allocation between GC-managed
heap space and I/O cache varies significantly across workloads and execution phases.

% \note{jk: in you approach you propose a resizing policy that takes into account
% GC and I/O costs. Explain what your system does. Then in other pargraph say that
% you implement this over TeraHeap. Then, say that you evaluate Lucene and write
% some high level results.}
In this work, we propose a Dynamic Heap Resizer, a system that continuously monitors both garbage collection 
(GC) and I/O costs to make runtime decisions on how to partition DRAM between the managed heap (H1) and the 
page cache for the secondary heap (H2). Our approach dynamically resizes H1 by increasing or decreasing its 
allocation based on observed GC pause times, concurrent GC CPU overhead, and I/O wait times. By doing so, it 
balances memory allocation to minimize overall application stall time. The resizer employs a lightweight finite 
state machine (FSM) policy that introduces wait states to observe the effects of each resizing action before taking 
further decisions, ensuring stability while remaining responsive to rapid workload changes. We implement the Dynamic 
Heap Resizer on top of TeraHeap, extending its hybrid heap framework with adaptive, runtime resizing capabilities without
requiring any application modifications.

We evaluate our system on Apache Lucene, a widely used full-text search engine, under memory-constrained scenarios using 
realistic query workloads with varied frequency and size characteristics. Our results show that the Dynamic Heap Resizer 
outperforms both hand-tuned and even DRAM division configurations. Specifically, it maintains low GC pause times during 
memory-intensive phases while ensuring sufficient page cache capacity for I/O-heavy phases, leading to improved end-to-end 
execution time, reduced tail latencies, and enhanced performance stability across diverse workloads.
