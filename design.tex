\section{Design}

In order for FlexHeap to effectively resize 
the heap and give the appropriate amount of DRAM to the page 
cache, it must accurately detect when performance bottlenecks occur
from garbage collection (GC) or I/O. 
In this section, we discuss FlexHeap's FSM for resizing actions,
describe the different types of GC overhead scenarios and how they 
are measured and explain how FlexHeap operates on sampling intervals to collect GC and I/O metrics.
\subsection{Interval-Based Measurement Model}

An interval is a custom measurement that is defined as the duration between two consecutive calls to the \texttt{dram\_repartition()} function.
\texttt{Dram\_repartition()} a FlexHeap function, responsible for comparing GC and I/O overheads
and making DRAM resizing decisions accordingly. 
This function is inserted at key GC safepoints within the G1 garbage collector, specifically at the end of the following phases: 

\begin{itemize}
  \item \texttt{G1CollectedHeap::do\allowbreak\_collection\allowbreak\_pause\allowbreak\_at\allowbreak\_safepoint\allowbreak\_helper()} for young and mixed collections,
  \item \texttt{ConcurrentMark::remark()} for the end of the concurrent marking cycle,
  \item \texttt{G1FullCollector::collect()} for full garbage collections.
\end{itemize}

Since G1 dynamically chooses which collection path to execute based on the application's needs, only one of these safepoint 
locations is reached per interval. As such, the interval begins immediately after a call to \texttt{dram\_repartition()} 
and ends when \texttt{dram\_repartition()} is next invoked.

Each interval therefore includes two components: a period of mutator (application) thread execution 
and any GC activity that occurs during that time. At the end of the interval, runtime statistics are 
collected and used to evaluate and adjust the 
DRAM partitioning accordingly for the next interval.


\subsection{Calculation of GC overheads}
In G1, garbage collection overhead comes from three primary sources:

\begin{itemize}
  \item \textbf{Measuring Stop-the-world (STW) pauses:} In these events
  all application threads (mutators) are paused so that the JVM can exclusively perform 
  GC-related work. In order to calculate the time overheads produced by G1’s stop-the-world (STW) phases, we measure the time spent inside key G1 safepoint functions. 
  Specifically, we place timers around functions such as \texttt{ConcurrentMark::remark()}, which is part of the concurrent marking cycle and performs the final reachability check to ensure that all live objects are identified before resuming execution. 
  We also track the duration of \texttt{G1CollectedHeap::do\allowbreak\_collection\allowbreak\_pause\allowbreak\_at\allowbreak\_safepoint\allowbreak\_helper()}, which encapsulates the STW pause during young and mixed collections, as well as \texttt{G1FullCollector::collect()}, which initiates a full GC event. 
  Finally, \texttt{G1CollectedHeap::prepare\_heap\_for\_mutators()} marks the end of GC and the point where application (mutator) threads resume.

  Not all of these functions are invoked in each and every interval, we monitor only those that are executed by the G1 GC path during the given interval. 

  To translate this time into CPU overhead, we use the following formula:
  \[
  \text{cpu\_time\_spent} += \text{gc\_pause\_time} \times \min(\text{mutators}, \text{cores})
  \]
  This formula estimates the compute time lost due to GC pauses by considering both the duration of the pauses and the level of available parallelism. 
  Although mutator threads are inactive during STW events, they represent potential application work that is delayed. 
  By only counting the threads that could actually run at the same time, we get a more accurate and realistic estimate of how much the application was slowed down. 

  \item \textbf{Measuring Concurrent GC threads:} These threads operate in the background,
  concurrently with application execution. 
  Hence, in order to account for compute cycles lost to concurrent GC threads, 
  we must determine whether GC activity actually interferes with 
  application (mutator) execution. This interference occurs only when 
  the system is oversubscribed—that is, when the combined number of 
  mutator threads and concurrent GC threads exceeds the number of available 
  CPU cores, resulting in context switching. In such cases, concurrent GC threads compete with the application 
  for CPU time, causing actual slowdown.

  If the system is not oversubscribed meaning that the number of mutator and GC
  threads together is less or equal than the number of cores, then the concurrent 
  GC threads are able to run on otherwise idle CPUs. In this case, we assume that
  CPUs remaining idle do not cause performance loss.

  Even in oversubscribed conditions, we avoid attributing all GC time as overhead.
  We first subtract the amount of time GC threads could have used on idle cores estimated 
  as the number of free cores \texttt{\#cores} $-$ \texttt{\#mutators} $x$ \texttt{interval\_duration} .
  The remaining time is treated as actual interference with the
  application and is counted as GC overhead.

  This logic is summarized as:

  \begin{center}
  \textit{GC overhead} = GC thread time $-$ time spent on idle cores (if oversubscribed), else $0$
  \end{center}

  \item \textbf{Measuring Refinement threads:} Refinement threads are a part of concurrent collection, 
  so the full runtime of refinement threads is interpreted as GC overhead. This is because refinement thread 
  activity directly supports concurrent GC phases and typically runs in the background,
  contributing 100\% of their execution time to GC-related compute loss.

\end{itemize}

To measure garbage collection overhead, our system tracks the cumulative CPU time spent in each of these three components. 
This data is collected in each interval and used by FlexHeap to determine whether GC-related costs are increasing or decreasing.

\subsection{Heap resizing Finate State Machine}

After FlexHeap collects the GC and I/O overheads of any given interval, it passes the metrics through 
a \textit{finite state machine (FSM)}. The FSM serves as a control mechanism 
that reacts to both GC and I/O overhead percent changes and decides whether to adjust the DRAM allocation 
between the heap (H1) and the page-cache. The FSM is evaluated at the end of every interval with a
call to the \texttt{dram\_repartition()} function. It consists of 3 states:

Firstly, the \texttt{wait\_after\_grow} state represents a phase that occurs after 
the system has increased the size of the heap (H1). Its purpose is to 
to monitor whether the previous growth decision effectively 
reduced overall overheads. 

Each time the state is evaluated, FlexHeap retrieves runtime data from the G1 collector, 
such as the current heap capacity, used and unused regions in bytes and the concurrent marking 
start threshold (IHOP), to determine whether the heap is approaching 
a GC initiation and whether there is available space to grow further.

The decision logic follows several key checks:
\begin{itemize}
  \item If the total overhead has increased compared to the previous interval, the system
    compares the relative growth rates of GC and I/O time. If GC overhead increased more than I/O, 
    FlexHeap interprets this as insufficient heap space and issues a \texttt{GROW\_HEAP} action, 
    provided the heap has not reached its maximum capacity. 
  \item Conversely, if I/O overhead increased more than GC and the heap utilization is below 95\% of
    total capacity, the system transitions to the \texttt{wait\_after\_shrink} state and issues a 
    \texttt{SHRINK\_HEAP} action, transferring DRAM to the page cache.
  \item If none of these conditions are met, the system goes to a \texttt{no\_action\_state}.
\end{itemize}

Also, IHOP acts as a safety condition. If the heap usage falls below the 
IHOP limit and the current capacity is under the maximum allowed size, the FSM remains in the 
grow state waiting to see if the system will stabilize, without issuing any further grow actions.

The second state is the \texttt{wait\_after\_shrink}. Its purpose is to 
observe how the system behaves after freeing DRAM back to the page cache. During this
state, the FSM compares the garbage collection (GC) with the I/O to determine whether
the shrink operation had reduced I/O pressure or if further resizing actions are required.

The core logic of this state is as follows:
\begin{itemize}
  % \item If the total overhead (\texttt{gc\_time\_ms + io\_time\_ms}) remains nearly unchanged
  % from the previous interval (within a 5\% margin) and the heap usage is below the IHOP threshold, 
  % the system transitions to the \texttt{no\_action} state. This indicates that the system has stabilized 
  % after shrinking and no further changes are necessary.
  
  \item 
  If GC overhead increased more than I/O, it indicates that shrinking the heap too aggressively,
  led to increased GC activity. In this case, the system transitions to the \texttt{wait\_after\_grow} state
  and performs a \texttt{GROW\_HEAP} action to give memory to the heap.
  
  \item If I/O overhead increased  the system stays in the \texttt{wait\_after\_shrink} state and issues further shrinks.
  If heap usage is still high (above 90\%), the FSM may also decide to grow again to relieve GC pressure.
\end{itemize}

Additionally, the state monitors total resident memory (rss) and page cache usage.
If the sum of resident memory and page cache usage falls below 80\% 
of the configured DRAM limit, the FSM interprets is as (\texttt{IOSLACK}) and 
issues another \texttt{SHRINK\_HEAP} action to allocate more memory to the page cache. This allows 
to reclaim unused DRAM to relieve I/O when the system is under light memory pressure.

If none of these conditions are triggered, the FSM 
transitions to \texttt{no\_action}.

At last the \texttt{no\_action} state, which is the default state.
When the system enters this state, no DRAM resizing actions were performed.

At each interval, FlexHeap evaluates whether the total GC and I/O time (\texttt{gc\_time\_ms + io\_time\_ms}) has
changed. If the change is within a 5\% threshold, it is interpreted 
as noise or a stable condition and the system remains in the \texttt{no\_action} state.

If the combined overhead has increased notably and the system is in this state, the FSM analyzes which overhead (GC or I/O) is 
contributing more to the increase:
\begin{itemize}
  \item If GC overhead has increased more than I/O, this suggests that H1 may need a grow action. 
  In that case, FlexHeap transitions to the \texttt{wait\_after\_grow} state and issues a \texttt{GROW\_HEAP} action, 
  unless the heap has already reached its maximum capacity.

  \item If I/O overhead increased more, FlexHeap checks if heap usage is above 90\% of 
  its capacity. If not, the system transitions to \texttt{wait\_after\_shrink} and issues a \texttt{SHRINK\_HEAP} 
  action to return DRAM to the page cache.
\end{itemize}

If the total overhead has decreased compared to the previous interval, no resizing action is needed. In this case, 
the system remains in the \texttt{no\_action} state.

\subsection{Heap Resizing Policies}

When a grow action is issued, FlexHeap invokes the \texttt{young\_collection\_expand\_amount()} function from 
the \texttt{G1HeapSizingPolicy} class to calculate how much the heap should be expanded.

The method begins by calculating the total amount of uncommitted memory which is the difference
between the reserved heap size (\texttt{max\_capacity()}) and the currently committed size 
(\texttt{capacity()}). This value represents the available space for heap expansion.

A scale factor is then computed using a delta cpu usage between the previous intervals and is bounded between two thresholds: \texttt{1.3} 
(minimum growth) and \texttt{4.2} (maximum growth). These growth values represent a x-axis value to a sigmoid
function which returns a scaled number based on how aggressive the heap will be expanded.

After scaling, the function clamps the resulting value between two bounds: a minimum threshold of one heap region size 
(\texttt{HeapRegion::GrainBytes}) and a maximum of the remaining uncommitted memory. 
The final value is the resize amount that FlexHeap passes onto the G1-Collector to resize its heap.

If a FlexHeap determines a shrink action is needed, the \texttt{young\_collection\_shrink\_amount()} is called, calculating a shrink amount,
that will be applied to the heap. Unlike the expand function, 
which dynamically scales based on CPU usage deltas, the shrink amount has a fixed maximum scaling factor, ensuring that the
shrink actions are more aggressive, in order to provide as much memory to the page-cache at any given moment. 
This factor is defined by the parameter \texttt{G1ShrinkByPercentOfAvailable}, 
which defaults to \texttt{50}, resulting in a maximum scale factor of \texttt{0.5}, or 50\% of the free memory regions,
while also leaving space for sudden allocation bursts.

To calculate the amount of memory to shrink, the function first finds the number of currently free heap 
regions via \texttt{\_g1h->num\_free\_regions()}. This value represents memory that is already committed but 
not actively being used. It then multiplies this number by the size of a single heap region (\texttt{HeapRegion::GrainBytes})
and scales it by the shrink factor. The result is the total number of bytes for a heap shrinkage.

