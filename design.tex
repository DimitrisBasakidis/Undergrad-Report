\section{Design}

\subsection{Overview}

The goal of FlexHeap is to dynamically split DRAM between the primary heap (H1)
and the page cache used by TeraHeap’s secondary heap (H2), to reduce garbage collection (GC) 
and I/O overheads. It monitors runtime behavior in short intervals and uses a finite state machine 
(FSM) to decide when to grow or shrink the main heap. GC and I/O costs are translated into lost CPU time and then
compared to determine the optimal resizing action. Expansion decisions are scaled based on CPU 
usage and are bound between limits to prevent overly aggressive or conservative heap growth. Shrink operations are more aggressive, aiming to release memory back to the page cache 
by reclaiming up to 50\% of the currently free heap regions. This design allows FlexHeap to respond to workload changes and improve system performance dynamically.

\subsection{Interval-Based Measurement Model}

All resizing decisions are made at the end of an interval. An interval is defined as the time between 
two consecutive young garbage collections (Young GCs). These intervals are recurring 
measurement windows in which both the application threads (mutator) and the garbage collector (GC) can perform working tasks.

During an interval, we track runtime metrics including:
\begin{itemize}
  \item The total CPU time lost due to GC, including stop-the-world pauses, concurrent GC threads, and refinement threads.
  \item The I/O overhead introduced by page cache while perfoming accesses to TeraHeap's secondary heap (H2).
\end{itemize}

Due to intervals being short, they are ideal for constantly checking GC and I/O overheads 
without introducing significant delay or noise. The cumulative behavior across intervals is
then interpreted by a finite state machine (FSM), which issues grow or shrink 
actions based on the recorded runtime metrics.

\subsection{Calculation of GC overheads}

In this section we highlight the main sources of Garbage Collection overhead:

\begin{itemize}
  \item \textbf{Measuring Stop-the-world (STW) pauses:} In these events
  all application threads (mutators) are paused so that the JVM can exclusively perform 
  GC-related work. In order to calculate the time overheads produced by G1’s stop-the-world (STW) phases,
  we measure the time spent inside key G1 safepoint functions. 
  Specifically we place timers around safepoint phases of the G1 garbage collector. 
  These include the stop-the-world pause for young and mixed collections, the concurrent 
  marking phase where the reachability of live objects is finalized and the full GC path
  that reclaims the entire heap when necessary. 

  After obtaining the time spent in the safepoint functions, we translate it into CPU overhead, by using the following formula:
  \[
  \text{cpu\_time\_spent} += \text{gc\_pause\_time} \times \min(\text{mutators}, \text{cores})
  \]
  This formula estimates CPU cycles lost due to GC pauses by considering both the duration of the pauses and the level of available parallelism. 
  Although mutator threads are inactive during STW events, they represent potential application work that is delayed. 
  By only counting the threads that could actually run at the same time, we get a more accurate and realistic estimate of how much the application was slowed down. 

  \item \textbf{Measuring Concurrent GC threads:} These threads operate in the background,
  concurrently with application execution. 
  Hence, in order to account for compute cycles lost to concurrent GC threads, 
  we must determine whether GC activity actually interferes with 
  application (mutator) execution. This interference only occurs when the combined number of 
  mutator threads and concurrent GC threads exceeds the number of available 
  CPU cores. In such cases, concurrent GC threads compete with the application 
  for CPU time, causing actual slowdown.

  If the total mutator and GC threads is less or equal than the number of cores, the concurrent 
  GC threads are able to run on otherwise idle CPUs. In this case, we assume that
  CPUs remaining idle do not cause performance loss.

  Even under conditions where concurrent GC threads and mutator threads compete for CPU resources we avoid attributing all GC time as overhead.
  We first subtract the amount of time GC threads could have used on idle cores estimated 
  as the number of free cores \texttt{\#cores} $-$ \texttt{\#mutators} $x$ \texttt{interval\_duration} .
  The remaining time is treated as actual interference with the
  application and is counted as GC overhead.

  \item \textbf{Measuring Refinement threads:} Refinement threads are a part of concurrent collection, 
  so the full runtime of refinement threads is interpreted as GC overhead. This is because refinement thread 
  activity directly supports concurrent GC phases and typically runs in the background,
  contributing 100\% of their execution time to GC-related compute loss.

\end{itemize}

To measure garbage collection overhead, our system tracks the cumulative CPU time spent in each of these three components. 
This data is collected in each interval and used by FlexHeap to determine whether GC-related costs are increasing or decreasing.

\subsection{I/O Monitoring via eBPF Library}

To accurately capture the I/O overhead produced by page‑cache accesses 
and object transfers in the secondary heap (H2), we used an existing TeraHeap
eBPF‑based monitoring library. This library runs in kernel space and tracks events such as page faults and read‑and‑write
operations of mapped regions. Whenever an object accessed in H2 triggers a page‑cache miss or eviction, the library reports the 
data movement, which we convert into CPU‑equivalent lost cycles. 
By integrating this eBPF library, I/O stalls can be translated into equivalent 
CPU time lost, enabling a direct comparison with GC overheads during each interval.



\subsection{Heap resizing Finate State Machine}

Upon completion of an interval, FlexHeap collects the GC and I/O lost compute cycles and passes them through 
a \textit{finite state machine (FSM)}. The FSM serves as a control mechanism 
that reacts to both GC and I/O overhead percent changes and decides whether to adjust the DRAM allocation 
between the heap (H1) and the page-cache. The FSM consists of 3 states:

Firstly, the \texttt{wait\_after\_grow} state, which represents a phase that occurs after 
the system has increased the size of the heap (H1). Its purpose is to 
to monitor whether the previous growth decision effectively 
reduced overall overheads without destabilizing the system. 

Each time the state is evaluated, FlexHeap retrieves runtime data from the G1 collector, 
such as the current heap capacity, used and unused regions in bytes and the concurrent marking 
start threshold (IHOP), to determine whether the heap is approaching 
a GC initiation and whether there is available space for further expansions.

The decision logic follows several key checks:
\begin{itemize}
  \item If the total GC and I/O overhead has increased compared to the previous interval, the system
    compares the relative growth rates of GC and I/O time. If GC overhead increased more than I/O, 
    FlexHeap interprets this as insufficient heap space and issues a \texttt{GROW\_HEAP} action, 
    provided the heap has not reached its maximum capacity. 
  \item Conversely, if the I/O overhead increased more than the GC and the heap utilization is below 95\% of
    total capacity, the system transitions to the \texttt{wait\_after\_shrink} state and issues a 
    \texttt{SHRINK\_HEAP} action, shrinking the heap and transferring DRAM to the page cache via cgroups.
  \item If none of these conditions are met, the system goes to a \texttt{no\_action\_state}.
\end{itemize}

Also, the systems uses IHOP as a safety condition. If the heap usage falls below the 
IHOP limit and the current capacity is under the maximum allowed size, the FSM remains in the 
grow state waiting to see if the system will stabilize, without issuing any further grow actions.

The second state is the \texttt{wait\_after\_shrink}. Its purpose is to 
observe how the system behaves after a heap shrinkage wihich consequently frees DRAM back to the page cache. During this
state, the FSM compares the garbage collection (GC) with the I/O to determine whether
the shrink operation had reduced I/O pressure or, if further resizing actions are required.

The core logic of this state is as follows:
\begin{itemize}
  
  \item 
  If GC overhead have increased more than I/O, it indicates that shrinking the heap too aggressively,
  led to increased GC activity. In this case, the system transitions to the \texttt{wait\_after\_grow} state
  and performs a \texttt{GROW\_HEAP} action to give memory to the heap.
  
  \item If the I/O overhead increased the system stays in the \texttt{wait\_after\_shrink} state and issues further shrinks.
  If the heap usage is still high (above 90\%), the FSM may also decide to grow again to relieve GC pressure.
\end{itemize}

Additionally, the state monitors total resident memory (rss) and page cache usage.
If the sum of resident memory and page cache usage falls below 80\% 
of the configured DRAM limit, the FSM interprets is as (\texttt{IOSLACK}) and 
issues another \texttt{SHRINK\_HEAP} action to allocate more memory to the page cache. This allows 
to reclaim unused DRAM to relieve I/O when the system is under light memory pressure.

If none of these conditions are triggered, the FSM 
transitions to \texttt{no\_action}.

At last the FSM supports the \texttt{no\_action} state, which is the default state.
When the system enters this state, no DRAM resizing actions were performed.

At each interval, FlexHeap evaluates whether the total GC and I/O time has
changed. If the change is within a 5\% threshold, it is interpreted 
as noise or a stable condition and the system remains in the \texttt{no\_action} state.

If the combined overhead has increased notably and the system is in this state, the FSM analyzes which overhead (GC or I/O) is 
contributing more to the increase:
\begin{itemize}
  \item If the GC overhead has increased more than I/O, this suggests that H1 may need a grow action. 
  In that case, FlexHeap transitions to the \texttt{wait\_after\_grow} state and issues a \texttt{GROW\_HEAP} action, 
  unless the heap has already reached its maximum capacity.

  \item If the I/O overhead has increased more, FlexHeap checks if the heap usage is above 90\% of 
  its capacity. If not, the system transitions to \texttt{wait\_after\_shrink} and issues a \texttt{SHRINK\_HEAP} 
  action to return DRAM to the page cache.
\end{itemize}

If the total overhead has decreased compared to the previous interval, no resizing action is needed. In this case, 
the system remains in the \texttt{no\_action} state.
 
\subsection{Heap Resizing Step}

When an expansion is required, the system calculates, by using the vanilla-G1's resizing step, how much uncommitted
memory is available. This is the memory that has been reserved by the JVM but not yet used. 
A scaling factor is computed based on the change in CPU overhead compared to the previous 
measurement interval. This factor determines the aggressiveness of the heap expansion and 
is bounded to prevent overly conservative or excessively aggressive resizing. The final 
resize amount is then clamped within threshold values to ensure it stays within bounds, large enough to make 
a adjustments, but not to exhaust the remaining memory and create GC pressure.

In contrast, shrink decisions are more aggressive. Since returning 
DRAM to the operating system's page cache can significantly reduce I/O delays for TeraHeap
workloads, shrink operations prioritize releasing unused memory. The amount of memory to shrink 
is derived from the number of free heap regions—memory that has already been committed but is 
not actively in use. A fixed scaling factor (50\%), is then applied to determine how much 
of the free memory can be reclaimed, ensuring that the main heap retains sufficient headroom 
for sudden bursts of memory allocation without triggering excessive garbage collection.

% % interval is the time frame between 2 young gcs
% % overview section
% % how do i measure the gc overheads (CPU time spent on STW and CPU on concurrent and refinmenet threads)
% % how do we measure I/O
% % applying the smae resizing step as vanila g1 with different thresholds

