%%%%%%%%%%%%%%%%%%%%%%%%%%%%%%%%%%%%%%%%%%%%%%%%%%%%%%%%%%%%%%%%%%%%%%%%%%%%%%%%
% Template for USENIX papers.
%
% History:
%
% - TEMPLATE for Usenix papers, specifically to meet requirements of
%   USENIX '05. originally a template for producing IEEE-format
%   articles using LaTeX. written by Matthew Ward, CS Department,
%   Worcester Polytechnic Institute. adapted by David Beazley for his
%   excellent SWIG paper in Proceedings, Tcl 96. turned into a
%   smartass generic template by De Clarke, with thanks to both the
%   above pioneers. Use at your own risk. Complaints to /dev/null.
%   Make it two column with no page numbering, default is 10 point.
%
% - Munged by Fred Douglis <douglis@research.att.com> 10/97 to
%   separate the .sty file from the LaTeX source template, so that
%   people can more easily include the .sty file into an existing
%   document. Also changed to more closely follow the style guidelines
%   as represented by the Word sample file.
%
% - Note that since 2010, USENIX does not require endnotes. If you
%   want foot of page notes, don't include the endnotes package in the
%   usepackage command, below.
% - This version uses the latex2e styles, not the very ancient 2.09
%   stuff.
%
% - Updated July 2018: Text block size changed from 6.5" to 7"
%
% - Updated Dec 2018 for ATC'19:
%
%   * Revised text to pass HotCRP's auto-formatting check, with
%     hotcrp.settings.submission_form.body_font_size=10pt, and
%     hotcrp.settings.submission_form.line_height=12pt
%
%   * Switched from \endnote-s to \footnote-s to match Usenix's policy.
%
%   * \section* => \begin{abstract} ... \end{abstract}
%
%   * Make template self-contained in terms of bibtex entires, to allow
%     this file to be compiled. (And changing refs style to 'plain'.)
%
%   * Make template self-contained in terms of figures, to
%     allow this file to be compiled. 
%
%   * Added packages for hyperref, embedding fonts, and improving
%     appearance.
%   
%   * Removed outdated text.
%
%%%%%%%%%%%%%%%%%%%%%%%%%%%%%%%%%%%%%%%%%%%%%%%%%%%%%%%%%%%%%%%%%%%%%%%%%%%%%%%%

\documentclass[letterpaper,twocolumn, 10pt]{article}
\usepackage{usenix2019_v3}

% To be able to draw some self-contained figs
\usepackage{tikz}
\usepackage{amsmath}
\usepackage{graphicx}
\usepackage{color, soul}
\usepackage{multirow}
\usepackage{enumitem}
\usepackage{graphicx}
\usepackage{subcaption}
\usepackage{floatrow}
\usepackage{xspace}
\usepackage{hyperref}
\usepackage{breakurl}
\usepackage{svg}

\urlstyle{same}
\usepackage[symbol]{footmisc}
%% Break urls in hyphens
\def\UrlBreaks{\do\/\do-}

% Inlined bib file
\usepackage{filecontents}
\newcommand{\note}[1]{\hl{\textbf{#1}}}

%-------------------------------------------------------------------------------
\begin{document}
%-------------------------------------------------------------------------------

% don't want date printed
\date{}

% make title bold and 14 pt font (Latex default is non-bold, 16 pt)
\title{Dynamic Heap Resizing for Tiered Managed Heaps over Fast Storage Devices}

% Comment for Double blind

{\renewcommand{\thefootnote}{\fnsymbol{footnote}}

\author{Dimitris Basakidis, csd4960}


\maketitle
}

{\renewcommand{\thefootnote}{\arabic{footnote}}
\setcounter{footnote}{0}

\pagenumbering{gobble}

\begin{abstract}
	\note{jk: TeraHeap does not use remote memory. It uses fast storage devices.
		Also, TeraHeap does not divide DRAM in two components but it uses two
		heapHeap does not divide DRAM in two components but it uses two heaps. In the
		problem statement you have to show the effects of GC and I/O costs}
	TeraHeap is a dual heap memory management system
	designed to support tiered memory by partitioning the DRAM budget between two
	heaps: The fast heap (H1), which resides on-heap and the remote heap (H2),
	which is backed by a remote file and accessed through the operating system's
	page cache. \note{jk: I do not understand this sentence. Please write carefully. Each word counts!!!}This division is managed statically which can lead to memory
	pressure on H1 while H2 remains underutilized, while the excess memory could
	be transfered over to H1 reducing the pressure, or vice versa.

	In this thesis, we address this problem \note{jk: what problem??} by
	integrating a dynamic DRAM partitioning mechanism into TeraHeap. \note{jk:
		write in high level what does this dynamic resizing policy do.} We port
	FlexHeap, a resizing policy designed to monitor runtime
	memory behavior and adjust the DRAM split between H1 and H2 accordingly. By
	enabling adaptive heap resizing, we aim to improve memory utilization and
	optimize both garbage collection and I/O overheads, under dynamic workloads
	with shifting memory demands.\note{jk: add some high level results!!!}

\end{abstract}

\section{Introduction}

% \note{jk: you need two paragraphs in intro to describe the problem statement.
% 	What I read in the first paragraph is not relevant with the problem that your
% 	thesis solves. The problem is: 1. Data analytics and search engines running on JVM need large heaps.
% 	2. A promising solution for large heaps that do not increase the GC cost is dual-heap designes, such as TeraHeap. Explain how does TeraHeap work and explain the problem of DRAM division.
% 	3. Then put a paragraph describing in high level your solution and how does this solution work.
% 	4. write where you implement your solution and provide some high-level results.
% }

% Widely-used data analytics and search engines such as \textbf{Apache Lucene}
% \cite{klinaftakis2025thesis} and \textbf{Apache Spark} run over Java virtual
% machines (JVM). They require large heaps in order to be able to process large
% datasets. \note{jk: Such systems require to host large compute caches. For
% 	example, Lucene maintains a query cache to store frequently queries to avoid
% 	their recomputation. Similarly, Spark maintains on-heap cache to store
% 	intermediate results to avoid recomputation. However, large-heaps require large
% 	amount of DRAM but DRAM capacity scalining is limited. Also, large-heaps
% 	requires expensive GC scans and compactions}.

Widely-used data analytics and search engines such as Apache Lucene
\cite{klinaftakis2025thesis} and Apache Spark \cite{spark3.3} run over Java Virtual
Machines (JVM) and rely on large managed heaps. These systems depend heavily on in-memory compute caches
to reduce redundant computation. For instance, Lucene maintains a query cache to
store results of frequent queries, while Spark uses on-heap caching to retain 
intermediate computation results across stages. However, supporting such large heaps
requires substantial DRAM capacity, which does not scale \cite{mandelman2002dram, white2011dram}.
In addition, large heaps significantly increase garbage collection (GC) overhead,
as they require more frequent and costly GC cycles, including full-heap scans and compaction. 

A solution for large heaps that does not add overheads to the GC cost is
dual-heap designs, such as TeraHeap \cite{teraheap_asplos}. TeraHeap extends
G1, the default garbage collector of OpenJDK to use two heaps: a primary heap
(H1) in DRAM and a second high capacity heap (H2) memory-mapped over a fast
storage device which is acccessed though OS page cache. G1 scans and compacts
objects in H1, but avoid GC scans over H2. The DRAM division between the main
heap and the pagecache must happen at the beginning of an execution statically.

This static DRAM division presents a problem, as the system cannot adapt to 
execution phases with different memory and I/O demands. During GC-intensive phases,
applications require a larger heap to reduce collection frequency and pause times. 
Conversely, in I/O-heavy phases, a larger page cache is needed to minimize page evictions
and reduce reads from the H2 storage file. Without dynamic adjustment, the system 
risks underutilizing available DRAM and suffering from either excessive GC overhead or high I/O latency.

To demonstrate the limitations of static DRAM partitioning, Figure~\ref{fig:lucene_m1} presents two Lucene 
M1 benchmark runs, each constrained to a total DRAM budget of 8\,G. The page cache is shown as a red line, 
the heap capacity as a purple line, and the used heap as a blue line. The left-hand side illustrates a static
configuration with a 4\,G heap (H1) and a 4\,G page cache. Early in the run, the heap usage spikes, triggering 
frequent GC cycles. Over time, heap usage stabilizes at a lower level, suggesting that a portion of the DRAM could 
have been reallocated to the page cache without heavily impacting GC behavior.

The right-hand side of the figure shows the same benchmark executed under a dynamic heap partitioning policy. In this case, 
the heap and page-cache sizes are adjusted dynamically at runtime. Notably, the page cache is expanded 
during I/O-intensive phases, mitigating page evictions and reducing I/O pressure. This is reflected in the 
lower iowait observed in the bottom-right plot, compared to the static run. 
These results highlight the importance of dynamic DRAM partitioning in mixed GC and I/O workloads.

\begin{figure*}[!t]
    \centering

    % Left column (Benchmark + IOWait)
    \begin{minipage}[t]{0.49\textwidth}
        \centering
        \includegraphics[width=\linewidth]{fig/combined_memory_timeline_vanilla_g1.png}
        \vspace{0.5em}
        \includegraphics[width=\linewidth]{fig/iow_cpu_teraheap.png}
        \subcaption{M1 Lucene benchmark with 4\,G H1 and 4\,G page-cache}
        \label{fig:lucene-m1_th}
    \end{minipage}
    \hfill
    % Right column (Benchmark + IOWait)
    \begin{minipage}[t]{0.49\textwidth}
        \centering
        \includegraphics[width=\linewidth]{fig/flexheap_debug.png}
        \vspace{0.5em}
        \includegraphics[width=\linewidth]{fig/iow_cpu_flex.png}
        \subcaption{M1 Lucene benchmark with 6\,G H1 and 2\,G page-cache}
        \label{fig:lucene-m1_flex}
    \end{minipage}

    \caption{Execution timeline (top) and I/O wait (bottom) for two M1 Lucene runs under static and dynamic DRAM partitioning.}
    \label{fig:lucene_m1}
\end{figure*}

To address this limitation, we port FlexHeap \cite{flexheap} to our system, a dynamic
resizing policy that operates at runtime by dividing execution into sampling
intervals. During each interval, it tracks the number of CPU time consumed by Garbage-Collection (GC)
and the I/O overheads including I/O caused by accesses to
the (H2) memory-mapped file. At the end of each interval, it compares the
percent change between the two metrics, relative to the previous intervals. If
garbage collection overhead has increased more than I/O stalls, FlexHeap
signals G1 to grow the H1 heap to relieve GC pressure. Inversely, if I/O delays
have increased, a shrink heap action is invoked, shifting the remaining memory
towards the pagecache, via cgroups, to reduce evictions and hold more data from
the H2 file.

Across Lucene and Spark benchmarks, FlexHeap achieves 
execution time improvements of up to 70\% and 9\%, compared to 
TeraHeap with static DRAM partitioning. We implement FlexHeap as an extension 
to OpenJDK 17, extending TeraHeap.
In this thesis we make the following contributions:

\vspace{0.5em}
\begin{itemize}
\item We identify and analyze the limitations of static DRAM partitioning in a tiered managed heap 
  environment, highlighting how fixed memory allocations can not optimize performance 
  due to underutilized memory and I/O pressure.

\item We intergate FlexHeap, into TeraHeap \cite{melidonis_thesis, mairh_thesis}.

\item We evaluate FlexHeap on real-world Lucene and Spark workloads, showing consistent
  performance improvements and better DRAM utilization under certain memory budgets.

\end{itemize}

\section{Background}
In order for FlexHeap to effectively resize 
the heap and give the appropriate amount of DRAM to the page 
cache, it must accurately detect when performance bottlenecks occur
from garbage collection (GC) or I/O. 
In this section, we provide background on the 
internal mechanisms of G1, describe the different types of GC overhead scenarios and how they 
are measured, break down how TeraHeap introduces I/O through page cache accesses to the remote heap (H2) 
and explain how FlexHeap operates on sampling intervals to collect GC and I/O metrics.
\subsection{Interval-Based Measurement Model}

An interval is a custom measurement that is defined as the duration between two consecutive calls to the \texttt{dram\_repartition()} function.
The \texttt{dram\_repartition()} function is responsible for comparing GC and I/O overheads
and making DRAM resizing decisions accordingly. Although the internal logic of this decision-making is discussed in detail 
in the next section, it serves as the signal point between intervals. 
This function is inserted at key GC safepoints within the G1 garbage collector, specifically at the end of the following phases: 

\begin{itemize}
  \item \texttt{G1CollectedHeap::do\allowbreak\_collection\allowbreak\_pause\allowbreak\_at\allowbreak\_safepoint\allowbreak\_helper()} for young and mixed collections,
  \item \texttt{ConcurrentMark::remark()} for the end of the concurrent marking cycle,
  \item \texttt{G1FullCollector::collect()} for full garbage collections.
\end{itemize}

Since G1 dynamically chooses which collection path to execute based on the application's needs, only one of these safepoint 
locations is reached per interval. As such, the interval begins immediately after a call to \texttt{dram\_repartition()} 
and ends when \texttt{dram\_repartition()} is next invoked.

Each interval therefore includes two components: a period of mutator (application) thread execution 
and any GC activity that occurs during that time. At the end of the interval, runtime statistics are 
collected and used to evaluate and adjust the 
DRAM partitioning accordingly for the next interval.


\subsection{G1 mechanisms and calculation of GC overheads}
In G1, garbage collection overhead comes from three primary sources:

\begin{itemize}
  \item \textbf{Stop-the-world (STW) pauses:} These are safepoint events
  where all application threads (mutators) are paused so that the JVM can exclusively perform 
  GC-related work. In order to calculate the time overheads produced by G1’s stop-the-world (STW) phases, we measure the time spent inside key G1 safepoint functions. 
  Specifically, we place timers around functions such as \texttt{ConcurrentMark::remark()}, which is part of the concurrent marking cycle and performs the final reachability check to ensure that all live objects are identified before resuming execution. 
  We also track the duration of \texttt{G1CollectedHeap::do\allowbreak\_collection\allowbreak\_pause\allowbreak\_at\allowbreak\_safepoint\allowbreak\_helper()}, which encapsulates the STW pause during young and mixed collections, as well as \texttt{G1FullCollector::collect()}, which initiates a full GC. 
  Finally, \texttt{G1CollectedHeap::prepare\_heap\_for\_mutators()} marks the end of GC and the point where application (mutator) threads resume.

  Not all of these functions are invoked in each and every interval, we monitor only those that are executed by the G1 GC path during the given interval. 

  To translate this time into CPU overhead, we use the following formula:
  \[
  \text{cpu\_time\_spent} += \text{gc\_pause\_time} \times \min(\text{mutators}, \text{cores})
  \]
  This formula estimates the compute time lost due to GC pauses by considering both the duration of the pauses and the level of available parallelism. 
  Although mutator threads are inactive during STW events, they represent potential application work that is delayed. 
  By only counting the threads that could actually run at the same time, we get a more accurate and realistic estimate of how much the application was slowed down. 

  \item \textbf{Concurrent GC threads:} These threads operate in the background,
  concurrently with application execution. They are responsible for marking live 
  objects, reclaiming unreachable memory, and performing cleanup tasks such as region 
  classification. While these threads improve throughput by avoiding long pauses, they 
  still consume CPU cycles.
  To account for the CPU overhead caused by concurrent GC threads, we compute \texttt{conc\_gc\_thr\_cpu\_time} as follows:

  \[
  \texttt{conc\_gc\_thr\_cpu\_time} = 
  \begin{cases}
  \texttt{conc\_gc\_thr\_cpu\_time} - \max(0, (\texttt{\#cores} - \texttt{\#mutators}) \times \texttt{interval}) & \text{if }
  \texttt{\#mutators} + \texttt{\#gc\_conc\_threads} > \texttt{\#cores} \\
  0 & \text{otherwise}
  \end{cases}
  \]

  This logic captures the effect of CPU oversubscription. When the total number of mutator and concurrent GC threads exceeds the number of available physical cores, the concurrent GC threads compete with the application for CPU time. In this case, we assume that only the fraction of GC time that actually contended for CPU resources contributes to lost compute cycles. Therefore, we subtract the portion of GC thread time that would have occurred on otherwise idle cores.

  The term \texttt{(\#cores - \#mutators) × interval} estimates the maximum time GC threads could have used without interfering with mutators. If this value is positive, it is subtracted from the measured GC thread time. If all cores are already occupied by mutators, the full concurrent GC thread time is counted as overhead. When the system is not oversubscribed, we assume concurrent GC threads do not delay mutators and assign zero overhead.

  In practice, \texttt{conc\_gc\_thr\_cpu\_time} is updated based on whether a GC thread was interrupted. If uninterrupted, we accumulate time directly using:
  \[
  \texttt{conc\_gc\_thr\_cpu\_time} += \texttt{conc\_gc\_thr\_cpu\_total\_time[i]}
  \]
  If a thread was interrupted, we compute the time by comparing the current thread CPU time to the last recorded start time:
  \[
  \texttt{conc\_gc\_thr\_cpu\_time} += \texttt{conc\_gc\_thr\_cpu\_total\_time[i]} + (\texttt{current\_thread\_cpu\_time} - \texttt{conc\_gc\_thr\_cpu\_start\_time[i]})
  \]

  This mechanism ensures that only the effective GC thread activity that could interfere with application performance is captured as lost compute cycles.

  \item \textbf{Refinement threads:} G1 uses card tables and remembered sets to track 
  inter-region references. Refinement threads are responsible for processing these write
  barriers and maintaining the remembered sets, which are essential for incremental and 
  concurrent collection. These threads introduce additional CPU overhead and can become a
  bottleneck when the number of cross-region references is high.
\end{itemize}

To measure garbage collection overhead, our system tracks the cumulative CPU time spent in each of these three components. This data is collected periodically and used by FlexHeap to determine whether GC-related costs are increasing or decreasing across sampling intervals.


\section{Design}

\subsection{Interval-Based Measurement Model}

All resizing decisions are made at the end of an interval. An interval is defined as the time between 
two consecutive young garbage collections (Young GCs) in the G1 collector. These intervals are recurring 
measurement windows in which both the application threads (mutator) and the garbage collector (GC) can perform working tasks.

During an interval, we track runtime metrics including:
\begin{itemize}
  \item The total CPU time lost due to GC, including stop-the-world pauses, concurrent GC threads, and refinement threads.
  \item The I/O overhead introduced by page cache while perfoming accesses to TeraHeap's secondary heap (H2).
\end{itemize}

Due to intervals being short, they are ideal for constantly checking GC and I/O overheads 
without introducing significant delay or noise. The cumulative behavior across intervals is
then interpreted by a finite state machine (FSM), which issues grow or shrink 
actions based on the recorded runtime metrics.

\subsection{Overview}

% The goal of the design is to dynamically divisio the DRAM between the 
% primary heap (H1) the page-cache for the H2, based on the application's runtime behavior. 
% This is achieved through a feedback-driven mechanism that monitors performance and makes resizing decisions.
%
% At the end of each interval, the system invokes the \texttt{dram\_repartition()} function, which
% compares the total compute time lost to garbage collection (GC) and I/O operations to the previous interval.
% Based on the comparison the FSM issues resizing actions in order to reduce the overheads in future intervals.

% The \texttt{dram\_repartition()} function compares these two overheads to determine which source of 
% pressure is dominant. If GC overhead is higher, more DRAM is allocated to the on-heap region (H1). 
% If I/O pressure dominates, DRAM is released from H1 to allow more space for page cache and reduce evictions. 
% To prevent erratic or overly aggressive resizing, decisions are moderated by a finite state machine (FSM) that
% encodes cooldown periods and threshold conditions. This ensures that resizing actions are taken smoothly and 
% only when justified by sustained performance changes.
%
% In summary, our system continuously monitors runtime behavior, compares GC and I/O overheads at the end of
% each interval, and uses an FSM-guided policy to repartition DRAM between H1 and H2 in a responsive yet 
% controlled manner.
%


% In order for FlexHeap to effectively resize 
% the heap and give the appropriate amount of DRAM to the page 
% cache, it must accurately detect when performance bottlenecks occur
% from garbage collection (GC) or I/O. 
% In this section, we discuss FlexHeap's FSM for resizing actions,
% describe the different types of GC overhead scenarios and how they 
% are measured and explain how FlexHeap operates on sampling intervals to collect GC and I/O metrics.
%
%
\subsection{Calculation of GC overheads}
In G1, garbage collection overhead comes from three primary sources:

\begin{itemize}
  \item \textbf{Measuring Stop-the-world (STW) pauses:} In these events
  all application threads (mutators) are paused so that the JVM can exclusively perform 
  GC-related work. In order to calculate the time overheads produced by G1’s stop-the-world (STW) phases,
  we measure the time spent inside key G1 safepoint functions. 
  Specifically we place timers around safepoint phases of the G1 garbage collector. 
  These include the stop-the-world pause for young and mixed collections, the concurrent 
  marking phase where the reachability of live objects is finalized and the full GC path
  that reclaims the entire heap when necessary. 

  After obtaining the time spent in the safepoint functions, we translate it into CPU overhead, by using the following formula:
  \[
  \text{cpu\_time\_spent} += \text{gc\_pause\_time} \times \min(\text{mutators}, \text{cores})
  \]
  This formula estimates CPU cycles lost due to GC pauses by considering both the duration of the pauses and the level of available parallelism. 
  Although mutator threads are inactive during STW events, they represent potential application work that is delayed. 
  By only counting the threads that could actually run at the same time, we get a more accurate and realistic estimate of how much the application was slowed down. 

  \item \textbf{Measuring Concurrent GC threads:} These threads operate in the background,
  concurrently with application execution. 
  Hence, in order to account for compute cycles lost to concurrent GC threads, 
  we must determine whether GC activity actually interferes with 
  application (mutator) execution. This interference occurs only when 
  the system is oversubscribed. This occurs, when the combined number of 
  mutator threads and concurrent GC threads exceeds the number of available 
  CPU cores, due to context switching. In such cases, concurrent GC threads compete with the application 
  for CPU time, causing actual slowdown.

  If the system is not oversubscribed meaning that the number of mutator and GC
  threads together is less or equal than the number of cores, then the concurrent 
  GC threads are able to run on otherwise idle CPUs. In this case, we assume that
  CPUs remaining idle do not cause performance loss.

  Even in oversubscribed conditions, we avoid attributing all GC time as overhead.
  We first subtract the amount of time GC threads could have used on idle cores estimated 
  as the number of free cores \texttt{\#cores} $-$ \texttt{\#mutators} $x$ \texttt{interval\_duration} .
  The remaining time is treated as actual interference with the
  application and is counted as GC overhead.

  \item \textbf{Measuring Refinement threads:} Refinement threads are a part of concurrent collection, 
  so the full runtime of refinement threads is interpreted as GC overhead. This is because refinement thread 
  activity directly supports concurrent GC phases and typically runs in the background,
  contributing 100\% of their execution time to GC-related compute loss.

\end{itemize}

To measure garbage collection overhead, our system tracks the cumulative CPU tims spent in each of these three components. 
This data is collected in each interval and used by FlexHeap to determine whether GC-related costs are increasing or decreasing.

\subsection{I/O Monitoring via eBPF Library}

To accurately capture the I/O overhead produced by page‑cache accesses 
and object transfers in the secondary heap (H2), we used an existing 
eBPF‑based monitoring library. This library runs in kernel space and tracks events such as page faults and read‑and‑write
operations of mapped regions. Whenever an object accessed in H2 triggers a page‑cache miss or eviction, the library reports the 
data movement, which we convert into CPU‑equivalent lost cycles. 
By integrating this eBPF library, I/O stalls can be translated into equivalent 
CPU time lost, enabling a direct comparison with GC overheads during each interval.



\subsection{Heap resizing Finate State Machine}

After the end of an interval, FlexHeap collects the GC and I/O lost compute cycles and passes them through 
a \textit{finite state machine (FSM)}. The FSM serves as a control mechanism 
that reacts to both GC and I/O overhead percent changes and decides whether to adjust the DRAM allocation 
between the heap (H1) and the page-cache. The FSM consists of 3 states:

Firstly, the \texttt{wait\_after\_grow} state, which represents a phase that occurs after 
the system has increased the size of the heap (H1). Its purpose is to 
to monitor whether the previous growth decision effectively 
reduced overall overheads without destabilizing the system. 

Each time the state is evaluated, FlexHeap retrieves runtime data from the G1 collector, 
such as the current heap capacity, used and unused regions in bytes and the concurrent marking 
start threshold (IHOP), to determine whether the heap is approaching 
a GC initiation and whether there is available space for further expansions.

The decision logic follows several key checks:
\begin{itemize}
  \item If the total overhead has increased compared to the previous interval, the system
    compares the relative growth rates of GC and I/O time. If GC overhead increased more than I/O, 
    FlexHeap interprets this as insufficient heap space and issues a \texttt{GROW\_HEAP} action, 
    provided the heap has not reached its maximum capacity. 
  \item Conversely, if the I/O overhead increased more than the GC and the heap utilization is below 95\% of
    total capacity, the system transitions to the \texttt{wait\_after\_shrink} state and issues a 
    \texttt{SHRINK\_HEAP} action, shrinking the heap and transferring DRAM to the page cache via cgroups.
  \item If none of these conditions are met, the system goes to a \texttt{no\_action\_state}.
\end{itemize}

Also, the systems uses IHOP as a safety condition. If the heap usage falls below the 
IHOP limit and the current capacity is under the maximum allowed size, the FSM remains in the 
grow state waiting to see if the system will stabilize, without issuing any further grow actions.

The second state is the \texttt{wait\_after\_shrink}. Its purpose is to 
observe how the system behaves after a heap shrinkage wihich consequently frees DRAM back to the page cache. During this
state, the FSM compares the garbage collection (GC) with the I/O to determine whether
the shrink operation had reduced I/O pressure or, if further resizing actions are required.

The core logic of this state is as follows:
\begin{itemize}
  
  \item 
  If GC overhead have increased more than I/O, it indicates that shrinking the heap too aggressively,
  led to increased GC activity. In this case, the system transitions to the \texttt{wait\_after\_grow} state
  and performs a \texttt{GROW\_HEAP} action to give memory to the heap.
  
  \item If the I/O overhead increased the system stays in the \texttt{wait\_after\_shrink} state and issues further shrinks.
  If the heap usage is still high (above 90\%), the FSM may also decide to grow again to relieve GC pressure.
\end{itemize}

Additionally, the state monitors total resident memory (rss) and page cache usage.
If the sum of resident memory and page cache usage falls below 80\% 
of the configured DRAM limit, the FSM interprets is as (\texttt{IOSLACK}) and 
issues another \texttt{SHRINK\_HEAP} action to allocate more memory to the page cache. This allows 
to reclaim unused DRAM to relieve I/O when the system is under light memory pressure.

If none of these conditions are triggered, the FSM 
transitions to \texttt{no\_action}.

At last the FSM supports the \texttt{no\_action} state, which is the default state.
When the system enters this state, no DRAM resizing actions were performed.

At each interval, FlexHeap evaluates whether the total GC and I/O time (\texttt{gc\_time\_ms + io\_time\_ms}) has
changed. If the change is within a 5\% threshold, it is interpreted 
as noise or a stable condition and the system remains in the \texttt{no\_action} state.

If the combined overhead has increased notably and the system is in this state, the FSM analyzes which overhead (GC or I/O) is 
contributing more to the increase:
\begin{itemize}
  \item If the GC overhead has increased more than I/O, this suggests that H1 may need a grow action. 
  In that case, FlexHeap transitions to the \texttt{wait\_after\_grow} state and issues a \texttt{GROW\_HEAP} action, 
  unless the heap has already reached its maximum capacity.

  \item If the I/O overhead has increased more, FlexHeap checks if the heap usage is above 90\% of 
  its capacity. If not, the system transitions to \texttt{wait\_after\_shrink} and issues a \texttt{SHRINK\_HEAP} 
  action to return DRAM to the page cache.
\end{itemize}

If the total overhead has decreased compared to the previous interval, no resizing action is needed. In this case, 
the system remains in the \texttt{no\_action} state.
 
\subsection{Heap Resizing Step}

When a grow action is issued, FlexHeap invokes the \texttt{young\_collection\_expand\_amount()} function, 
which is a vanilla G1 resizing function.

The method begins by calculating the total amount of uncommitted memory which is the difference
between the reserved heap size (\texttt{max\_capacity()}) and the currently committed size 
(\texttt{capacity()}). This value represents the available space for heap expansion.

A scale factor is then computed using a delta cpu usage between the previous intervals and is bounded between two thresholds: \texttt{1.3} 
(minimum growth) and \texttt{4.2} (maximum growth). These growth values represent a x-axis value to a sigmoid
function which returns a scaled number based on how aggressive the heap will be expanded.

After scaling, the function clamps the resulting value between two bounds: a minimum threshold of one heap region size 
(\texttt{HeapRegion::GrainBytes}) and a maximum of the remaining uncommitted memory. 
The final value is the resize amount that FlexHeap passes onto the G1-Collector to resize its heap.

If a FlexHeap determines a shrink action is needed, the \texttt{young\_collection\_shrink\_amount()} is called, calculating a shrink amount,
that will be applied to the heap. Unlike the expand function, 
which dynamically scales based on CPU usage deltas, the shrink amount has a fixed maximum scaling factor, ensuring that the
shrink actions are more aggressive, in order to provide as much memory to the page-cache at any given moment. 
This factor is defined by the parameter \texttt{G1ShrinkByPercentOfAvailable}, 
which defaults to \texttt{50}, resulting in a maximum scale factor of \texttt{0.5}, or 50\% of the free memory regions,
while also leaving space for sudden allocation bursts.

To calculate the amount of memory to shrink, the function first finds the number of currently free heap 
regions via \texttt{\_g1h->num\_free\_regions()}. This value represents memory that is already committed but 
not actively being used. It then multiplies this number by the size of a single heap region (\texttt{HeapRegion::GrainBytes})
and scales it by the shrink factor. The result is the total number of bytes for a heap shrinkage.
%
% % interval is the time frame between 2 young gcs
% % overview section
% % how do i measure the gc overheads (CPU time spent on STW and CPU on concurrent and refinmenet threads)
% % how do we measure I/O
% % applying the smae resizing step as vanila g1 with different thresholds


\section{Methodology}
In our evaluation, we will be answering the following questions:

\begin{itemize}
\item How does Dynamic Heap Resizing improve performance compared to static DRAM partitioning?
\item What is the impact of dynamic resizing on garbage collection overhead and application execution time?
\item How robust is the Dynamic Heap Resizer under different lucene workloads?
\end{itemize}

All experiments were conducted on titan2, a dedicated server equipped with two Intel Xeon Gold 5318Y CPUs (48 physical cores, 96 hardware threads), 256 GB of DDR4 DRAM, and a 1.7 TB NVMe SSD used for the H2 file. The system runs Ubuntu 24.04.1 LTS with Linux kernel version 6.8.0-1020-nvidia. For each experiment the DRAM capacity has been limited by using cgroups.
We evaluate our system using a 300 GB real-world Lucene dataset containing 52.6 billion words. We preprocess this dataset by performing a complete word count and filtering out stop words, single-character words, and non-English words. Based on frequency distributions, words are categorized into three groups: high-frequency (top 1\% of words, accounting for ~80\% of total occurrences), medium-frequency (mid-range words), and low-frequency (91\% of words, contributing only ~1\% of occurrences). This classification models different levels of I/O load, as high-frequency words lead to significantly higher document access rates compared to low-frequency words.

We construct queries of two sizes: small queries retrieving the top 50 results and large queries retrieving up to 500,000 results. Large queries place higher pressure on the managed heap, while frequency classes primarily influence I/O cost. Our evaluation includes three workloads:
M1, high-frequency small queries (HS)
M2, high-frequency large queries (HL) and
M3, a mixed workload consisting of medium-frequency small (MS) and medium-frequency large (ML) queries, reflecting realistic deployments where both I/O load and heap pressure vary dynamically.

For baseline comparison, we evaluate three configurations. First, we run Lucene using TeraHeap with a static DRAM partition, allocating 50\% of the 40 GB DRAM budget to H1 (managed heap) and 50\% to H2 (page cache backed by NVMe storage), representing a typical balanced setup when no detailed tuning is performed. Second, we run Lucene with hand-tuned ideal configurations, where the DRAM allocation between H1 and H2 is manually optimized to minimize execution time and garbage collection overheads for each benchmark. Finally, we evaluate our Dynamic Heap Resizer, which automatically adjusts the size of H1 at runtime. This experimental setup enables us to compare the effectiveness of dynamic DRAM partitioning against fixed static configurations and manually optimized ideal baselines.
% For baseline comparison, we evaluate two configurations. First, we run Lucene using TeraHeap with a static DRAM partition, allocating 50\% of the 40 GB DRAM budget to H1 (managed heap) and 50\% to H2 (page cache backed by NVMe storage). This 50-50 configuration represents the commonly recommended default setup, as it provides a balanced trade-off without requiring detailed workload-specific tuning, and is therefore widely adopted in practice. Secondly, we evaluate our Dynamic Heap Resizer, which automatically adjusts the size of H1 at runtime based on observed GC and I/O costs. The Dynamic Heap Resizer is initially launched with a DRAM allocation of 95\% to H1 and only 5\% to the H2 page cache, allowing it to adaptively shrink or grow the managed heap as needed during execution. This experimental setup enables us to compare the effectiveness of dynamic DRAM partitioning against fixed static configurations and manually optimized ideal baselines

\section{Evaluation}
First, we investigate the performance of Dynamic Heap Resizer 
(DHR) compared to two static configurations: (1) hand-tuned 
(ideal) and (2) even DRAM division (50-50). 
Figure~\ref{fig:gc_exec_time} depicts the normalized 
execution time of three representative Lucene benchmarks 
(M1, M2, and M3) on the Titan2 server. Within each group,
the first and second columns correspond to static ideal and 
static 50-50 partitioning, respectively, while the third
column represents DHR.

Overall, DHR consistently outperforms both static
configurations across all benchmarks. In particular,
compared to the static 50-50 configuration (even), DHR
achieves execution time reductions of 14.3\% for M1, 15.8\% 
for M2, and 1.3\% for M3. Compared to the hand-tuned (ideal)
configurations, DHR provides performance improvements of 
up to 13.4\% (M2). 
Although DHR improves overall performance, we observe an increase 
in GC time, particularly in M1 and M2. This increase is attributed
to high minor GC activity caused by the creation of many backward 
pointers within these workloads. Backward pointers are references 
from older to younger objects, which prevent objects from being 
collected during minor GCs, resulting in higher promotion rates and
increased GC overhead.

Specifically, GC time nearly doubles in M1 (1.87–1.93× increase) and
increases by approximately 19\% (1.18×) in M2 compared to static configurations. 
In contrast, M3 exhibits a slight decrease in GC time, indicating minimal backward 
pointer impact.


\begin{figure}[htbp]
  \centering
  \includegraphics[width=1\columnwidth]{fig/eval_graph.png}
  \caption{Execution time and GC time comparisons}
  \label{fig:gc_exec_time}
\end{figure}


% \section{Related Work}
\note{jk: the related work seems that is copy paste based on what I gave you.
This is not correct. You have to write your own text}
Prior approaches fall into four categories: cached heaps, tiered heaps, hybrid
heaps, and managed heap resizing techniques.

Cached heaps like Semeru \cite{semeru}, MemLiner \cite{memliner}, and Mako
\cite{mako} allocate the managed heap entirely on remote memory while using
local DRAM as a cache. These systems often require kernel modifications to
implement remote paging and suffer from high I/O traffic and GC overheads due to
frequent object scans and evacuations between memory tiers. For example, Semeru
offloads GC scans to remote JVM instances to mitigate latency, while MemLiner
reorganizes GC thread memory access order to align with mutator access patterns.
However, both still suffer from costly object transfers and rewriting between
memory tiers. Mako further introduces a Heap Indirection Table to track object
locations, which imposes load reference barriers on every access, adding CPU
overhead. Friendly-NVM-GC \cite{friendlynvmgc} uses DRAM as a cache for a heap
entirely placed on NVM but cannot avoid slow GC scans over NVM-resident objects.

Tiered heaps aim to reduce write amplification or NVM access latencies. Crystal
Gazer \cite{crystalgazer1, crystalgazer2} and GCMove \cite{gcmove} frequently
migrate objects between DRAM and NVM generations to limit writes but incur slow
scans to update references. ThinGC \cite{thingc} ensures mutators never directly
access NVM-resident objects by moving them into DRAM upon access, avoiding
immediate reference updates through lazy barriers, but this increases DRAM
memory pressure and GC frequency. JPDHeap \cite{jpdheap} requires programmers to
annotate objects for placement across DRAM and NVM, again leading to expensive
NVM scans for reference adjustments.

Hybrid heaps such as Melt \cite{melt} and LeakSurvivor \cite{leaksurvivor} place
hot objects in DRAM and cold objects on HDD or SSD, maintaining a user-space
cache inside the JVM to track object offsets. While effective for HDD latency,
they require a cache lookup for every access, adding CPU management overhead.
TeraHeap \cite{TeraHeap}, on the other hand, targets NVMe SSDs with
memory-mapped I/O to reduce cache lookup costs, but its DRAM partition between
the heap (H1) and page cache for the secondary heap (H2) is statically
configured and does not adapt to workload changes.

Finally, managed heap resizing techniques focus on improving GC performance by
dynamically adjusting heap sizes. Brecht et al. \cite{brecht} propose aggressive
heap expansion in Boehm GC to reduce execution time, while Yang et al.
\cite{yang} develop an analytical model for heap resizing in multi-tenant
environments. ElasticMem \cite{elasticmem} reclaims heap memory for other
processes in shared clusters to increase throughput. White et al. \cite{white}
use PID controllers to meet user-defined GC goals by tuning heap resize ratios.
However, these approaches solely optimize GC behavior without considering I/O
costs, limiting their applicability to hybrid heaps that require balanced DRAM
allocation between managed memory and page cache for performance stability.

In contrast to these systems, our Dynamic Heap Resizer monitors runtime GC
overhead and I/O cost, enabling dynamic partitioning of DRAM within hybrid
managed heaps.

\section{Conclusions}
As shown in \cite{smith2020example}, modern systems require secure storage.

\input{ack}
\bibliographystyle{plain}
\bibliography{paper}

%-------------------------------------------------------------------------------
\end{document}
%-------------------------------------------------------------------------------
