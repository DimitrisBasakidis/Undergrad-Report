\begin{abstract}
	% \note{jk: TeraHeap does not use remote memory. It uses fast storage devices.
	% 	Also, TeraHeap does not divide DRAM in two components but it uses two
	% 	heapHeap does not divide DRAM in two components but it uses two heaps. In the
	% 	problem statement you have to show the effects of GC and I/O costs}
	% TeraHeap is a dual heap memory management system
	% designed to support tiered memory by partitioning the DRAM budget between two
	% heaps: The fast heap (H1), which resides on-heap and the remote heap (H2),
	% which is backed by a remote file and accessed through the operating system's
	% page cache. \note{jk: I do not understand this sentence. Please write carefully. Each word counts!!!}This division is managed statically which can lead to memory
	% pressure on H1 while H2 remains underutilized, while the excess memory could
	% be transfered over to H1 reducing the pressure, or vice versa.
	%
	% In this thesis, we address this problem \note{jk: what problem??} by
	% integrating a dynamic DRAM partitioning mechanism into TeraHeap. \note{jk:
	% 	write in high level what does this dynamic resizing policy do.} We port
	% FlexHeap, a resizing policy designed to monitor runtime
	% memory behavior and adjust the DRAM split between H1 and H2 accordingly. By
	% enabling adaptive heap resizing, we aim to improve memory utilization and
	% optimize both garbage collection and I/O overheads, under dynamic workloads
	% with shifting memory demands.\note{jk: add some high level results!!!

	% corrected


  TeraHeap is a memory management system supporting a dual heap design: the primary heap
  (H1), which resides in DRAM, and the secondary heap (H2), which is memory-mapped over 
  a storage device and accessed via the OS page cache. The memory partitioning
  between H1 and the page cache for H2 is statically configured at the start of execution. 
  This static allocation fails to account for the dynamic nature of application behavior. Real-world 
  workloads often go through distinct phases where the relative pressure of garbage collection 
  (GC) and I/O varies. For instance, some phases may generate many short-lived objects, 
  stressing GC, while others may involve scans, on the H2 heap, that increase I/O. Without the 
  ability to adapt to these shifts, a static configuration can lead to suboptimal performance due to
  underutilized DRAM.

  In this thesis, we address the limitations of TeraHeap's static DRAM partitioning by 
  integrating FlexHeap, a dynamic heap resizing mechanism. 
  FlexHeap monitors garbage collection and I/O activity at runtime, compares their estimated costs 
  and adjusts the size of the H1 heap accordingly, assigning the remaining memory 
  to the page cache used for H2 accesses. In our evaluation,
	we show that FlexHeap improves the execution time of Lucene's benchmarks up
	to 70\%, while also reducing GC time and H2 file reads on average by over 35\% and
	55\%. Similarly, it reduces the total runtime in the majority of Spark's
	benchmarks, by 6\% on average.


\end{abstract}
