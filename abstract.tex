\begin{abstract}
	% \note{jk: TeraHeap does not use remote memory. It uses fast storage devices.
	% 	Also, TeraHeap does not divide DRAM in two components but it uses two
	% 	heapHeap does not divide DRAM in two components but it uses two heaps. In the
	% 	problem statement you have to show the effects of GC and I/O costs}
	% TeraHeap is a dual heap memory management system
	% designed to support tiered memory by partitioning the DRAM budget between two
	% heaps: The fast heap (H1), which resides on-heap and the remote heap (H2),
	% which is backed by a remote file and accessed through the operating system's
	% page cache. \note{jk: I do not understand this sentence. Please write carefully. Each word counts!!!}This division is managed statically which can lead to memory
	% pressure on H1 while H2 remains underutilized, while the excess memory could
	% be transfered over to H1 reducing the pressure, or vice versa.
	%
	% In this thesis, we address this problem \note{jk: what problem??} by
	% integrating a dynamic DRAM partitioning mechanism into TeraHeap. \note{jk:
	% 	write in high level what does this dynamic resizing policy do.} We port
	% FlexHeap, a resizing policy designed to monitor runtime
	% memory behavior and adjust the DRAM split between H1 and H2 accordingly. By
	% enabling adaptive heap resizing, we aim to improve memory utilization and
	% optimize both garbage collection and I/O overheads, under dynamic workloads
	% with shifting memory demands.\note{jk: add some high level results!!!

  % corrected

  TeraHeap is a memory management system supporting a dual heap design:
  the heap (H1), and the (H2) which is memory-mapped over a storage device
  and is accessed via the OS page cache. In order to be decided, how much memory goes into 
  H1 and the page-cache, it has to be statically configured, in the beginning of an execution.
  This leaves the system without adaptability during execution. H1 may need more memory, when it cannot
  get it, GC runs more often and for longer, increasing pause times and delaying the application.
  Conversely, when the page cache is smaller than it could be, storage traffic rises by increasing I/O wait,
  page evictions, and the system repeatedly reads from the H2 file.

  In this thesis, we address the lack of adaptability in TeraHeap’s static DRAM split, by integrating  
  FlexHeap, a dynamic heap resizing mechanism into TeraHeap. FlexHeap is, a resizing policy that monitors 
  at runtime GC and I/O. Based on these metrics it resizes the Garbage Collector's heap accordingly
  and assigns the rest of the DRAM budget to the page-cache via cgroups. In our evaluation, we show that FlexHeap
  improves the execution time of Lucene's benchmarks up to 70\%, while also reducing GC time and H2 file reads 
  on average by 28\% and 13\%.
   

  \end{abstract}
