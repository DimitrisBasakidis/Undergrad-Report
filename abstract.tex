\begin{abstract}
\noindent
% \note{jk: Fix the problem statement. The problem here is that DRAM cannot scale so systems like TeraHeap
% use fast storage devices to extend their heap. Then you have to say that these
% systems to avoid large GC pause time they use two heaps: a primary heap over DRAM and
% a second heap over storage device. Say that they use I/O cache to reduce slow
% accesses to the storage device. These systems thy need to divide dynamically the
% DRAM between H1 and I/O cache for H2.}
Managed runtimes must efficiently balance memory allocation across dynamically changing workloads, 
especially under constrained DRAM environments. As DRAM capacity scaling has slowed, systems like
TeraHeap extend the Java heap over fast storage devices such as NVMe SSDs to handle large data volumes.
To minimize garbage collection (GC) overheads while leveraging storage capacity, teraHeap employs a 
tiered heap design with two regions: H1, a primary heap for GC-managed short-lived objects
and H2, a secondary heap mapped to storage for long-lived data. To accelerate accesses 
to H2 and avoid frequent slow I/O operations, a portion of DRAM is also used as a page cache for H2.

However, current systems such as TeraHeap statically divide DRAM between H1 and the page cache for H2, 
lacking dynamic adaptability to changes in application memory pressure or I/O demands. In this work, 
we extend TeraHeap with a Dynamic Heap Resizer that adjusts the DRAM partitioning between H1 and the 
H2 cache at runtime. Our approach monitors total lost compute cycles, combining GC-induced CPU contention 
and I/O wait delays, to decide whether to grow or shrink H1 dynamically. The resizer uses a lightweight policy 
engine fully integrated into the OpenJDK’s G1 TeraHeap implementation and requires no manual tuning or application-specific knowledge.

We evaluate our system on Lucene, a widely used full-text search engine, under varied memory-constrained scenarios.
Results show that the Dynamic Heap Resizer significantly improves adaptability and performance stability across workloads. 
It efficiently reclaims memory during idle periods to enhance system utilization and reallocates heap space under load to maintain 
throughput, outperforming ideal TeraHeap configurations under constrained DRAM in both responsiveness and efficiency.
\end{abstract}
