\begin{abstract}
	\note{jk: TeraHeap does not use remote memory. It uses fast storage devices.
		Also, TeraHeap does not divide DRAM in two components but it uses two
		heapHeap does not divide DRAM in two components but it uses two heaps. In the
		problem statement you have to show the effects of GC and I/O costs}
	TeraHeap is a dual heap memory management system
	designed to support tiered memory by partitioning the DRAM budget between two
	heaps: The fast heap (H1), which resides on-heap and the remote heap (H2),
	which is backed by a remote file and accessed through the operating system's
	page cache. \note{jk: I do not understand this sentence. Please write carefully. Each word counts!!!}This division is managed statically which can lead to memory
	pressure on H1 while H2 remains underutilized, while the excess memory could
	be transfered over to H1 reducing the pressure, or vice versa.

	In this thesis, we address this problem \note{jk: what problem??} by
	integrating a dynamic DRAM partitioning mechanism into TeraHeap. \note{jk:
		write in high level what does this dynamic resizing policy do.} We port
	FlexHeap, a resizing policy designed to monitor runtime
	memory behavior and adjust the DRAM split between H1 and H2 accordingly. By
	enabling adaptive heap resizing, we aim to improve memory utilization and
	optimize both garbage collection and I/O overheads, under dynamic workloads
	with shifting memory demands.\note{jk: add some high level results!!!}

\end{abstract}
