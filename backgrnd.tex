\section{Background}

\subsection{Page Cache I/O in TeraHeap}


TeraHeap extends the JVM by introducing a secondary heap, 
(H2), which is on a file in a remote device and accesses are passed through 
the operating system’s page cache. When marked objects get 
transferred from the on-heap memory (H1) to the H2 file, 
in order to be accessed later, they must first be brought back 
into memory through the page cache. 

This design uses the page cache to manage access to H2, 
but causes I/O when data is loaded from or written to disk. Each time an object 
stored in H2 is accessed, the corresponding pages are loaded. When 
the page cache becomes full, these loads trigger evictions of previously existing pages, which in 
turn generate read and write operations to the H2 file. Also when the data 
size that is brought to the page-cache exceeds the available capacity, repeated evictions and reloads occur, 
producing increasead I/O. This behavior constitutes the primary source of I/O 
overhead in TeraHeap.

Although additional I/O operations such as log writing, file reads, print/debug outputs
may also occur during the execution of applications, they are typically considered secondary and contribute much less to 
overall system I/O. 

