\section{Background}

\note{jk: Dimitri I want you to think what you are writing... What does it mean
	Page Cache I/O ? it does not make sense. Your intention of this section is to provide a background on how TeraHeap works.}

\subsection{Page Cache I/O in TeraHeap}


TeraHeap extends the JVM by introducing a secondary heap,
(H2), which is on a file \note{jk: are you using a remote device on another
	node??? the device is local.. please be carefull on what your writing.} in a
remote device and accesses are passed through
the operating system’s page cache. \note{jk: Explain that the second heap is
	memory-mapped over the storage device. this is how you access the second
	heap}When marked objects get transferred from the on-heap memory (H1) to the H2
file, in order to be accessed later, they must first be brought back
into memory through the page cache.

This design uses the page cache to manage access to H2, but causes I/O when
data is loaded from or written to disk. Each time an object stored in H2 is
accessed, the corresponding pages are loaded. When the page cache becomes full,
these loads trigger evictions of previously existing pages, which in turn
generate read and write operations to the H2 file. Also when the data size that
is brought to the page-cache exceeds the available capacity, repeated evictions
and reloads occur, producing increasead I/O. This behavior constitutes the
primary source of I/O overhead in TeraHeap.

Although additional I/O operations such as log writing, file reads, print/debug
outputs may also occur during the execution of applications, they are typically
considered secondary and contribute much less to overall system I/O.

