\section{Background}


\subsection{Page Cache I/O in TeraHeap}


TeraHeap extends the JVM by introducing a secondary heap (H2), is a memory-mapped file,
on a storage device, for less frequenlty accessed objects.
Objects can be marked and then transferred, from G1-Collectors's heap (H1), to the H2 file.
To access these objects later, the data must first be fetched back through the page cache.

This design causes I/O when data is loaded from or written. Each time an object stored in H2 is
accessed, the corresponding pages are loaded, in page segments. When the page cache becomes full,
these loads trigger evictions of previously existing pages, which in turn
generate read and write operations to the H2 file. Also when the data size 
brought into the page-cache exceeds the available capacity, repeated evictions
and reloads occur, producing increasead I/O. This behavior constitutes the
primary source of I/O overhead in TeraHeap.

Although additional I/O operations such as log writing, file reads, print/debug
outputs may also occur during the execution of applications, they are typically
considered secondary and contribute much less to overall system I/O.

\subsection{GC Types in G1-Garbage Collector}
In the G1 garbage collector, GC overheads occur from 3 key
mechanisms that operate either at safepoints or concurrently with the application.
These contributors to GC-related CPU overhead are:

\textbf{1. Stop-the-World (STW) Pauses:}  
G1 performs certain tasks such as root scanning, object copying, or final marking and heap related operations
like young generation collections, mixed collections and full GCs. Such phases are exeuted in safepoints
where all application (mutator) threads are suspended. These are called \emph{stop-the-world} events and
directly block application execution resulting in major latency spikes.

\textbf{2. Concurrent GC Threads:}  
To reduce pause times, G1 assigns specific GC work to dedicated concurrent threads that run 
alongside the executors. These threads are responsible for concurrent markings.
While these operations are designed to be background activities,
they still consume CPU cycles, due to context switches. Also, if the number of GC threads plus mutator threads exceeds the available 
CPU cores, they compete for CPU time, leading to potential application slowdowns due to oversubscription.

\textbf{3. Refinement Threads:}  
G1 uses a card table and remembered sets to track cross-region references. Refinement threads 
are responsible for updating these remembered sets with references to objects existing in the old generation. 
This mechanism ensures correctness during evacuation but introduces additional CPU overhead, especially under write-heavy workloads.


